\documentclass{article}
\usepackage[utf8]{inputenc}

\title{GU-Makro-Formler}
\author{orts }
\date{April 2016}

\usepackage{natbib}
\usepackage{graphicx}

\begin{document}

\maketitle

\section{Introduction}
Formler för makroteori. Går igenom och förklarar alla relevanta formler. 



\section{Conclusion}
``I always thought something was fundamentally wrong with the universe'' \citep{adams1995hitchhiker}

\section{Långa sikten F2-F11}
\title{Föreläsning 1}

Under krisen 2008 så sjönk BNP per capita kraftigt. Idag har den återhämtat sig så att den är på samma nivå som 2007. 

$$
    \Delta \ ln \ Z = \Delta \ln  X + \Delta \ ln Y - \Delta \ ln Q
$$
Förenkling vid förändring. Gäller vid följande. 
$$
Z = \frac{X*Y}{Q}
$$

Idag har Sverige överskott i bytesbalansen. 

Det existerar tre sätt att bestäma värdet som produceras inom ett lands gränser. 

\begin{enumerate}
    \item Användning. Summa utgifter för nya varor/tjänster för slutgiltig användnign minus import. 
    \item Produktion: summa förädlingsvärde.
    \item Inkomst: summa kaptial- och arbetsinkomster. 
\end{enumerate}

\begin{equation}
    BNP = \ Y\ = \ C + I + C^G + I^G + X - IM 
\end{equation}

Bruttonationalinkomsten, BNI: 

$$
BNI = Y + Y^F \ (Y^F: landets \ faktorinkomster.) 
$$
Faktorinkomster: ersättning till arbete och kapitel som "ägs" av landets medborgare. Det vill säga avkastning från utlandet. 

Dissponibel buruttonationalinkomst, DBNI: 

$$
DBNI = Y + Y^F + Tr^F 
$$


\textbf{Bytesbalansen}: Denna summerar flödet mellan oss och omvärlen. 

$$
Bytesbalansen = NX + Y^F + Tr^F
$$
Har vi överskott i vår bytesbalans så kan detta investeras, dvs lånas ut till andra länder. Om detta är negativt så måste vi låna från de länder som har överskott. 


Nominell BNP = löpande priser.
Real BNP = satt mot ett basår. 
\par \noindent
Tillväxttakt är $g_{t+1}$

\vspace{5mm}


\title{Föreläsning 2}
\vspace{5mm}\par

Sveriges tillväxt startade i mitten på 1800-talet. 

\textbf{Kort sikt < 2år}: Löner och priser är fasta. 

\textbf{Lång sikt 4-5år}: Löner och priser anpassar sig till jämviktsvärden. 

På kort sikt så är konjukturläget det intressanta: inflation, ränta, arbetslöshet, stabeliseringspolitik och budgettunderskott. 

\begin{enumerate}
 \item MPK = kaptialets marginalprodukt. 
 \item MPL = arbetets marginalprodukt. 
\end{enumerate}

\textbf{Prdoutktionsfunktionens egenskaper:}
\begin{enumerate}
 \item Marginalprodukterna är positiva
 \item Marginalprodukterna är avtagande
 \item Konstant skalavkastning
\end{enumerate}

Produktionsfunktion: $$ Y = F(K,N) $$
$$ Y = K^{\alpha}N^{1-\alpha}$$

E = teknisk nivå. 

$$ Y = F(K,EN)$$

\textbf{Jämviktssysselsättning (natrulig)}:
$$N^n = (1-u^n)L $$

L = arbetskraft. 

\begin{itemize}
    \item Långsiktig tillväxt är viktigare än konjukturläget för genomsnittlig inkomst.
    \item Produktionsfunktionen visar producerad kvantitet för given teknisk nivå, kaptialstock, och antal arbetare. 
\end{itemize}

\par
 
 \vspace{5mm}
 \title{Föreläsning 3}
 \vspace{5mm}
 
\par \noindent
Köpkraft = $\frac{W}{P}$, nominell lön delat med prisnivån.\par \noindent
Inverterad efterfrågefunktion: $P_i = P(Y_i,P,Y)$ \vspace{5mm} \par \noindent

Priselasticitet: $$\frac{P_i*dY_i}{dP_i*Y_i}$$

Det genomsnittliga priset: 
$$
P = (1-\mu)MC = (1-\mu)\frac{W}{MPL}
$$

\textbf{Sluten ekonomi: Inkomst = produktion}
Det är köpkraften hos befolkningen som är det viktiga. Detta medför följande: 

$$
 \frac{W}{P} = \frac{MPL}{1+\mu}
$$
Om konkurrensen mellan företag ökar så minskar $ \mu$. \par \noindent

\textbf{Löneandel}: Denna är total lönesumma / nominell BNP 

$$
\frac{WN}{PY} = \frac{1-\alpha}{1+\mu}
$$

På lång sikt så når kapital per arbetare $ \frac{K}{EN} = k $ ett stabilt läge, steady state. 

\textbf{Komihåg till tenta att förklara alla modeller och formler!}

Löneandelen har legat relativt konstant under flera decennier. På ca 1/3. 
Lönesprdningen har ökat, detta kan knytas till automatiseringen. 

\vspace{5mm}
\title{Föreläsning 4} \par \noindent
\vspace{5mm}

$i_t$ är den nominella räntan mellan två perioder. Inflation är hur mycket priset förändras. Det vill säga förändringen i pris från en period till en annan. 
\textbf{Inflation: }
$$
\pi_{t+1} = \frac{P_{t+1}-P_t}{P_t}
$$

Relaränta är ungefär: $r_{t+1} \approx i_t - \pi_{t+1}$

Netto investering: 
$$
K_{t+1} - K_t = I_t -\delta K_t 
$$

\textbf{Optimala kaptialstcken: }
$$
\frac{MPK_{t+1}}{1+\mu}-\delta = r_{t+1}
$$
Ovanstående ekvation säger oss att; Det vi tjänar på att utöka kaptialstocken minus förtäringen på kapitalstocken måste vara lika med priset vi förväntas att betala för det. 


\textbf{Investeringsfunktion:} $I = I(Y^e,K,r)$

\vspace{5mm}
\title{Föreläsning 5}
\vspace{5mm}
\par \noindent

\textbf{Konsumption:} Denna bestäms av inkomsten idag, konsumtionen idag och inkomst i nästa period. 

$$
C_2 = Y^l_2 + (1+r)(Y^l_1-C_1)
$$

$$
\frac{U'(C1_)}{U'(C_2)} = \frac{1+r}{1+\rho}
$$
Den subjektiva räntan $ \rho $ bestämmer hur vida personen annser att det är mer värdefullt att spendera pengar i denna perioden jämfört med nästa. 

\textbf{Konsumptionfunktion:} $ C = C(Y,Y^e,r,A)$

Om den reala räntan ökar så vill befolkningen öka sitt sparande också, då du får mer för dina sparade pengar. Med ökad ränta så sjunker konsumtion och investeringar. 

\includegraphics[scale=0.4]{skarm2}

\vspace{5mm}
\title{Föreläsning 7}
\vspace{5mm}
\par \noindent

Den optimala kapitalstocken är också villkor för den långsiktiga jämviktsnivån. 
Genom att ta fram Kaptialets marginalprodukt och sätta in så får vi ut $K^*$.  Detta på lång sikt. På kort sikt kan vi endast kolla på hur det förändrar $MPK \ och \ MPL$. 
Produktionen på lång sikt tar vi fram på liknande vis. Den ges av: 

$$
Y^* = (K^*)^{\alpha}(EN)^{1-\alpha}
$$

Den långsiktiga realräntan = subjektiva diskonteringsräntan.
\par
En sluten ekonomi med konstant befolkning och teknisk nivå. $ \Rightarrow $ varken produktion eller konsumtion kan växa på lång sikt. \par

I en långsiktig jämvikt har vi ingen tillväxt. 

Från ekvationen: 
$$
Y^* = (\frac{\alpha}{(\rho+\delta)(1+\mu)})^{\frac{\alpha}{1-\alpha}}EN
$$

Så här kan det se ut över tid om kaptialstocken eller liknande minskar kraftigt, snabbt: 


\includegraphics[scale=0.6]{skarm3}

\textbf{Befolkningsökning och teknisk utveckling} \par \noindent Den tillväxt som sker i befolkning och teknisk utveckling beskrivs med $ n = \frac{\Delta N}{N} \ g =  \frac{\Delta E}{E} $. BNP växer i takt med befolkning och teknsik utveckling. Detta på lång sikt. Med tillväxt så förväntar sig konsumenterna en högre inkomst imorgon, de spenderar mer idag, men alla kan inte låna mer i en sluten ekonomi. \par \noindent Konsumenterna måste "mutas" att inte spendera mer idag. \par

\textbf{Algoritm för att bestämma långsiktig BNP/capita: }
\begin{enumerate}
 \item Bestäm $ y = \frac{Y}{EN} $
 \item Sätt in med optimal kapitalstock, ta ut $ k^* $
 \item Sätt in den ekvationen som fås för $ k^* $ i orginal ekvationen för $ y = \frac{Y}{EN} $ som ger $ Y^* = (k^*)^{\alpha}EN $
\end{enumerate}


\vspace{5mm}
\title{Föreläsning 8}
\vspace{5mm} \par \noindent 

Vi kan se tydliga skillnader mellan Nord- och Sydkorea, de ena är öppet och fritt medan de andra stängt och planekonomi. Syd är 20ggr rikare idag än Nord. Detta är direkt beroende på vilken typ av ekonomi. 

\begin{itemize}
    \item Privat kapital: maskiner, byggnader
    \item Offentligt kapital: infrastruktur
\end{itemize}

\textbf{Inkomstskillnader: } Dessa kan försklaras av, fysiskt kapital ca $ 20 $, humankapital mellan $ 10  \ till \ 30  $  resterande måste bero på något annat, E ,[procent].
Då teknologin i fattiga länder många gånger är väldigt låg så är deras BNP också lägre. De kan helt enkelt inte mäta sig med omvärlden på grund utav deras dåliga tillgång till teknik. 

\vspace{5mm}
\title{Föreläsning 9}
\vspace{5mm} \par \noindent 

\textbf{Arbetslöshet:} \par \noindent Frågan är här; Varför söker inte unga jobb? Är de för få platser? Är det för fås som slutar? Kompetenser varierar mellan personer, så gör också viljan att ha ett jobb.  \par \noindent  OLF: de är dem som står utanför arbetskraften, sjuka, studerande, och/eller pensionärer.  \par \noindent Frågan är; Hur stor är chansen att få jobb om du söker?  \par \noindent 
Andelen som söker jobb delat med den andel jobb som öppnas upp. 
$$
f \approx \frac{s}{\lambda u+s}
$$
Där \textbf{s} är den andel som som slutar varje månad.  \par \noindent 

\begin{itemize}
    \item $ \lambda$: Färre sökande per plats om de sökande är mindre aktiva.
    \item s: 3-4 ggr vanligare att byta jobb i Tyskland och USA.
    \item u: Fler som söker $ \Rightarrow $ svårare att får jobb. 
\end{itemize}

Chansen att få jobb är större då fler lämnar och/eller när låg arbetslöshet gäller. Arbetslöshet är ett problem för både samhälle och individen. Ekonomiska kostnader och psyko-social konsekvenser. Det är framförallt den långsiktiga arbetslösheten som är svår, denna kan leda till utanförskap, tappad kompetens, och lägre inkomst. Sverige finns många starka unioner (fackföreningar) dessa medför en högre arbetslöshet. Dessa har inte lika stor påverkan i andra länder så som USA. 

\begin{itemize}
    \item $\lambda f $ är andelen arbetslösa som får jobb under månaden.
    \item v är andelen arbetslösa som lämnar arbetskraften. 
\end{itemize}

x är andelen som lämnar arbetslösheten varje månad. Denna blir mindre varje månad. Till slut så resutlerar den i en geometrisk serie som ger den andra ekvationen.  

$$
 x = \lambda f + v 
$$
$$
\frac{1}{\lambda f + v}
$$

\vspace{5mm}
\title{Föreläsning 10}
\vspace{5mm} \par \noindent 

\textbf{Teorier om arbetslöshet:}
\vspace{5mm} \par \noindent Effektivitets-löneteorin handlar om hur företag sätter löner för att minimera kostnaden för ett givet antal anställda. Om det är låg arbetslöshet så har företagen incitament att höja löner. Detta då det finns få som söker jobben. Vice versa gäller vid hög arbetslöshet, då omvänt. 

\includegraphics[scale=0.5]{skarm4}

Den önskade lönen blir där med: 
$$
W^d = (1+a+bu)W
$$
Vad skull ske på lång sikt om $ u < u^n$? \par \noindent 
Detta skulle medföra att $ N > N^n $. Det blir svårt att behålla och rekrytera personal. Se vad som händer på grafen. 

\includegraphics[scale = 0.5]{skarm5} \par

\textbf{Sökfriktioner} handlar om att arbetare har olika kvalitéer och preferenser. Det tar också tid att hitta rätt person till rätt jobb. De faktorer som påverkar graden av friktion och arbetslösheten är: \textbf{Intensiteten}, hur mycket och ofta de arbetslösa söker jobb. \textbf{Kräsenhet}, hur kräsna de är med vilket jobb de får, \textbf{Konkurrens}, hur väl de kan konkurera med andra på marknaden. Skulle inte sökfriktioner existera så skulle detta medföra att kurvan för $ u^n $ låg på en lägre nivå. Då sökfriktionerna medför att vissa inte får jobb, eller inte lika lätt i alla fall. Det vill säga sökfriktioner ökar på den naturliga arbetslösheten. Det finns visa vktiga faktorer för att påverka sökfriktioner. Dessa är sådan som: regler för arbetslöshetsersättning, utbildningssystmets förmåga att utrusta arbetskraft med färdigheter, och arbetsmarknadspolitikens utformning (arbetsförmedling osv). \par \noindent \vspace{5mm} \textbf{Appendix:} \par \noindent 
I Sverige existerar inte några minimelöner. Dock existerar det lägsta löner i de flesta kollektivavtal. Minimelöner kan medföra arbetslöshet speciellt för lågkvalificerade grupper. Den teknologiska utvecklingen kan vara en förklarande faktor till detta. Detta ger samma effekt som sökfriktioner. Större andel ställs utanför arbetsstyrkan. Länder med en generös arbetslöshetsersättning har högre arbetslöshet, detta kan bero på att de arbetslösa finner sig i att vara det. Därmed blir lata/bekväma. 

\vspace{5mm}
\title{Föreläsning 11}
\vspace{5mm} \par \noindent 

\textbf{Inflation och teori} \par \noindent \vspace{5mm}

Vad bestämmer efterfrågan på pengar? 
$$
M*V = P*Y 
$$
V: Velocity, hur ofta du får betalt, i princip. 
$$
M^d = \frac{1}{V} PY
$$
På lång sikt så kan vi se att prisnivån blir den övre formeln fast med substituerat $ i$. Inflationen om $r^n, \pi, V$ är konstant. Ger oss med hjälp av förenkling gjord i förelsäning 1 följande formel. 

$$ 
Y = Y^n: \ \pi = \frac{\Delta P }{P} = \frac{\Delta M}{M} - \frac{\Delta Y^n}{Y^n}
$$

Slutsatsen blir att om tillgången på pengar växer snabbare än jämviktsproduktionen så resulterar detta i inflation. Om landet försöker att finansiera offentliga utgifter till stor del med hjälp av att trycka mer pengar, det vill säga öka den monetära basen, så kommer det att resultera i en hyperinflation. Detta var vad som skede med Zimbabwe och Tyskland.  \par \noindent \vspace{5mm} 

\vspace{5mm}
\title{Föreläsning 12}
\vspace{5mm} \par \noindent 

\textbf{IS-LM}
\vspace{5mm} \par \noindent 

IS-LM modellen har med Y som består av konsumtion och investeringar, samt realränta: $ r = i-\pi^e$. LM delen är funktionen för hur efterfrågan för pengar ser ut. IS ekvationen får vi från den avreagerade efterfrågan och den totala inkomsten. Den kan ses i nedanstående graf. Den långsiktiga inkomsten är helt vertikal, det vill säga beror inte alls i längden på efterfrågan. 

\includegraphics[scale=0.5]{skarm6} \vspace{5mm} \par \noindent 

\textbf{Multipliceringseffekten:} Den säger att vid en viss ändring $ \Delta I$ i aggregerad efterfrågan så sker en större ändring i $ Y$. En ökning i ränta medför en sänkning av konsumtion och investeringar. Detta kan ses i nedanstående bild. 

\includegraphics[scale=0.5]{skarm7}


\bibliographystyle{plain}
\bibliography{references}

\end{document}
